% SOMMARIO

\chapter*{Sommario}\label{ch:sommario}
Il quadrirotore è la piattaforma aerea maggiormente utilizzata ai fini di ricerca sui velivoli autonomi per la sue caratteristiche dinamiche che gli conferiscono un'ottima manovrabilità. Inoltre, sebbene sia instabile a ciclo aperto (come la maggior parte dei velivoli ad ala rotante), il quadrirotore mostra un buon grado di disaccoppiamento, cioè l'azione su uno dei segnali di controllo influisce quasi esclusivamente sul corrispondente grado di libertà.\\

La crescente complessità dei sistemi di controllo per questo tipo di mezzi necessita di modelli matematici estremamente accurati per la progettazione e la simulazione di relative leggi di controllo.\\

Lo scopo della seguente tesi è duplice. Il primo è quello di presentare un modello matematico che descriva rigorosamente la dinamica di un quadrirotore. Il secondo è quello di identificare un modello sufficientemente accurato per via sperimentale secondo un approccio a scatola nera, cioè esclusivamente sulla conoscenza di dati input-output. Nello specifico, sfruttando la proprietà di disaccoppiamento, si cercherà di identificare delle funzioni di trasferimento tempo continue in grado di descrivere il movimento del drone longitudinale, laterale e verticale (cioè lungo gli assi principali).\\

\textbf{Parole Chiave} \hspace{3mm} \textit{Unmanned Aerial Vehicle, Quadcopter, Quaternion, Robot Operating System, Gazebo, ArduCopter, MAVROS, System Identification, Black Box}
