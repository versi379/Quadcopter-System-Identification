% CHAPTER 1

\chapter{Introduzione}\label{ch:intro}
Nell'ultimo decennio l'industria robotica è cresciuta anno dopo anno, assumendo un ruolo decisivo nel più ampio campo dell'industria tecnologica. Secondo il rapporto "State of the robotics market" condotto da ABI Research \cite{ABI}, nel 2018 l'investimento totale nel settore della robotica è stato di circa 4 miliardi di dollari (US\$).\\

In particolare, la robotica mobile rappresenta uno dei trend maggiormente in crescita, con i veicoli a guida automatica - \ac{AGV} - che ne rappresentano un ruolo chiave nel mercato. Questi ultimi sono utilizzati principalmente in campo industriale per la movimentazione di prodotti all'interno di uno stabilimento. Tra le aziende leader di questo mercato primeggia Amazon, con investimenti continui volti a migliorare la produttività dei propri centri di distribuzione.\\

ABI sostiene che nel prossimo futuro gli \acs{AGV} continueranno ad avere un ruolo chiave nel settore della robotica mobile, ma prevede che i robot mobili autonomi - \ac{AMR} - li supereranno (in numero) entro il 2030. Questa ultima tipologia di robot mobili è caratterizzata da automi in grado di comprendere e spostarsi nell'ambiente esterno senza essere gestiti direttamente da un operatore.\\

In questo contesto, gli aeromobili a pilotaggio remoto - \ac{UAV} o droni - rappresentano un settore in via di sviluppo e trovano riscontro in un numero crescente di applicazioni: dal settore militare al settore civile, dal settore commerciale al settore dell'intrattenimento.

% NOTA: AGGIUNGERE INFO/ESEMPI AGV, AMR, UAV

\pagebreak

\section*{Breve Sommario}
L'elaborato si articola nei seguenti cinque capitoli:
\begin{itemize}
	\item \textbf{Capitolo 1} introduce la robotica mobile e le relative classificazioni
	\item \textbf{Capitolo 2} espone nel dettaglio la classe dei droni (\acs{UAV}), quindi descrive il modello fisico-matematico e il controllo di un quadrirotore 
	\item \textbf{Capitolo 3} introduce \acs{ROS} e ne spiega il funzionamento entrando nel dettaglio dei vari livelli che lo costituiscono, quindi descrive l'architettura utilizzata per simulare il volo di un quadrirotore
	\item \textbf{Capitolo 4} riporta le simulazioni di volo effettuate e i relativi dati input-output registrati, quindi spiega il procedimento seguito per ottenere modelli matematici \acs{SISO} che legano gli ingressi e le uscite fortemente dipendenti tra loro (e.g. rollio e posizione lungo y, beccheggio e posizione lungo x, spinta e posizione lungo z)
	\item \textbf{Capitolo 5} discute i risultati ottenuti e presenta eventuali sviluppi futuri
\end{itemize}
